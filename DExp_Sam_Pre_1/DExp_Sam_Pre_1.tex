%!TEX program = xelatex

% 制作请参考:https://github.com/tuna/THU-Beamer-Theme/blob/master/slide.tex
% 以及:https://www.zhihu.com/question/29676847

% ---------------------------------------------------------------------
% 基本设定
\documentclass[aspectratio=169]{beamer}
\usepackage{ctex}
\usepackage[T1]{fontenc}

\input{beamer_YsySettings}

\captionsetup[figure]{name=Figure} % 图片形式
\captionsetup[table]{name=Table} % 表格形式 



%---------------------------------------------------------------------
%	正文
%---------------------------------------------------------------------



\begin{document}
%---------------------------------------------------------------------
%---------------------------------------------------------------------
%---------------------------------------------------------------------		
	
	
	% 个人信息
	\author{戴鹏辉 \& 杨舒云}
	\title{基于热电效应的热机设计与热力学第二定律验证实验}
	\subtitle{设计性实验“热力学第二定律”开题报告}
	\date{May 16, 2024}
	%---
	
	
	% 封面
	\frame[plain]{\titlepage}
	% ---
	
	
	% 总目录
	\begin{frame}
		\frametitle{Outline}
		\tableofcontents
	\end{frame}
	% ---
	
	
	% 第一章
	\section{回顾实验要求}
	
	% 章目录
	\frame{\frametitle{Outline}\tableofcontents[currentsection]}
	
	% 内容
	\begin{frame}
		\frametitle{回顾实验要求I}
		
		\begin{block}{目的}
			设计并实现输出功率在1瓦以上的“热机”,以探究和验证热力学第二定律。
		\end{block}
		
	\end{frame}	
	
	\begin{frame}
		\frametitle{回顾实验要II}
		
		\begin{block}{要求}
			\begin{enumerate}
				\item \textbf{学习和理解热电效应}:
				\begin{itemize}
					\scriptsize \item 包括Seebeck效应、Peltier效应和Thomson效应。
					\item \scriptsize 设计实验方案,包含原理和物理模型。
				\end{itemize}
				
				\item \textbf{制作热机}:
				\begin{itemize}
					\scriptsize \item \scriptsize 展示热力学第二定律的“热机”。
					\item \scriptsize 电或机械输出功率不低于1瓦。
				\end{itemize}
				
				\item \textbf{测量与分析}:
				\begin{itemize}
					\scriptsize \item \scriptsize 测量装置的最大输出功率和输出效率。
					\item \scriptsize 讨论实际结果与Carnot循环的差异。
					\item \scriptsize 探讨进一步提高效率的方法。
					\item \scriptsize 分析测量精度和不确定度。
				\end{itemize}
				
				\item \textbf{确保安全性}:
				\begin{itemize}
					\scriptsize \item \scriptsize 确保装置表面(可触摸到的部分)温度不高于50\degree C。
				\end{itemize}
			\end{enumerate}
		\end{block}
	\end{frame}
	
	\begin{frame}
		\frametitle{回顾实验要求III}
		
		\begin{block}{熟悉基本实验装置,搭建与操作}
			\begin{itemize}
				\item \textbf{开路输出电压与温差的关系}:
				\begin{itemize}
					\item Seebeck效应,器件的等效热电系数 $\alpha$ 和等效热导 $\lambda$(热阻 $\rho$)。
				\end{itemize}
				
				\item \textbf{特定负载下输出功率与温差的关系}:
				\begin{itemize}
					\item 验证热力学第二定律。
				\end{itemize}
				
				\item \textbf{热源功率和单片热电堆输出效率的优化}:
				\begin{itemize}
					\item 在给定热源功率下,优化室温条件下单片热电堆的输出效率。
				\end{itemize}
				
				\item \textbf{固定冷、热源温度下的测量}:
				\begin{itemize}
					\item 测量在不同负载下的热电机输出功率与输出效率的关系。
					\item 分析器件内阻 $r$。
				\end{itemize}
			\end{itemize}
		\end{block}
	\end{frame}
	
	% ---
	
	
	% 第二章
	\section{实验原理概述}
	
	% 章目录
	\frame{\frametitle{Outline}\tableofcontents[currentsection]}
	
	% 内容
	\begin{frame}
		\frametitle{实验原理概述:三种热电效应I}
		
		\begin{block}{Seebeck效应}
			\small 当两种不同的导体或半导体连接成回路,并且两个接头的温度不同,就会在回路中产生\textcolor{c4}{电动势}。
			
			\begin{figure}[htbp]
				\centering
				\includegraphics[width=0.5\textwidth]{SamPre_1_Gra1.jpg}
			\end{figure}
			
			\begin{myhighlight}
				$$\mathrm{d}V=\epsilon_{AB}\mathrm{d}T$$
			\end{myhighlight}
			\footnotesize 其中,$\epsilon_{AB}$是温差电动势系数(又称\textcolor{c4}{Seebeck系数},记为$\alpha$)。符号约定为如果在高温段电动势驱使电流由金属A流向金属B为正。			
		\end{block}
		
	\end{frame}
	
	\begin{frame}
		\frametitle{实验原理概述:三种热电效应II}
		
		\begin{block}{Peltier效应}
			\small 当电流通过两种不同材料组成的电路时,一个接头会吸热,另一个接头会放热。这个效应对于\textcolor{c4}{调控热机的温度}非常重要,尤其是在确保装置表面温度不超过50\degree C的安全要求下。
			
			\begin{figure}[htbp]
				\centering
				\includegraphics[width=0.5\textwidth]{SamPre_1_Gra2.jpg}
			\end{figure}
			
			\begin{myhighlight}
				$$\textbf{J}_{q\pi}=\pi_{AB}\textbf{J}_{e}$$
			\end{myhighlight}
			\footnotesize 其中,$\textbf{J}_{q\pi}$是Peltier热流密度,$\textbf{J}_{e}$是从A到B的电流密度,$\pi_{AB}$是两种金属的Peltier系数(与温度有关)。
		\end{block}
		
	\end{frame}
	
	\begin{frame}
		\frametitle{实验原理概述:三种热电效应III}
		
		\begin{block}{Thomson效应}
			\scriptsize 在均质导体中,如果存在温度梯度,当电流通过时,会伴随着吸热或放热的现象。这对于完整的热电模型和效率分析很关键。$\dot{Q}=\mu I\cdot \nabla T$,$\mu$为Thomson系数。
		\end{block}
		
		\begin{block}{热电模型参数}
			\begin{itemize}
				\scriptsize \item \scriptsize 等效热导表示材料传导热量的能力,单位通常为瓦每米每开尔文(W/m·K)。等效热导越大,材料的热传导能力越强。
				$
				\lambda = \frac{Q}{A \cdot \Delta T \cdot t}
				$。
				\scriptsize 其中,$\lambda$ 是等效热导(W/m·K),$Q$ 是通过材料的热量(J),$A$ 是材料的横截面积($m^2$),$\Delta T$ 是材料两端的温差(K),$t$ 是热量传导所需的时间(s)。
				\scriptsize 通过Fourier热传导定律,也可以表示为(其中 $L$ 是材料的长度(m)):
				$
				Q = \lambda \cdot A \cdot \frac{\Delta T}{L} \cdot t
				$。
				
				\item \scriptsize 热阻表示材料对热流阻碍的能力,单位通常为开尔文每瓦(K/W)。热阻越大,材料的热流阻碍能力越强。
				$
				R_{\text{th}} = \frac{\Delta T}{Q}
				$。
				\scriptsize 其中,$R_{\text{th}}$ 是热阻(K/W),$\Delta T$ 是材料两端的温差(K),$Q$ 是通过材料的热量(W)。
				
				\item \scriptsize 等效热导和热阻是互为倒数的关系:
				$
				R_{\text{th}} = \frac{L}{\lambda \cdot A}
				$,
				$
				\lambda = \frac{L}{R_{\text{th}} \cdot A}
				$。
			\end{itemize}
		\end{block}
		
	\end{frame}
	
	\begin{frame}
		\frametitle{实验原理概述:输出电路部分}
		
		\begin{block}{Seebeck效应电源的外输出特性}
			\begin{itemize}
				\item 开路电压是指没有外部负载连接时,热电发电装置两端的电压。根据Seebeck效应,开路电压与温差成正比,$V_{\text{oc}} = \alpha \Delta T$。
				
				\item 当热电发电装置连接到负载时,输出电压会由于内阻的存在而下降。负载电压$V_{\text{L}} = \frac{\alpha \Delta T \cdot R_{\text{L}}}{R_{\text{L}} + R_{\text{in}}}$,而输出功率是负载上消耗的功率$P_{\text{out}} = \frac{(\alpha \Delta T)^2 \cdot R_{\text{L}}}{(R_{\text{L}} + R_{\text{in}})^2}$。
				
				\item \textcolor{c4}{当负载电阻等于内阻时},热电发电装置输出的功率达到最大$P_{\text{max}} = \frac{(\alpha \Delta T)^2}{4 R_{\text{in}}}$。				
			\end{itemize}						
		\end{block}
		
	\end{frame}
	
	\begin{frame}
		\frametitle{实验原理概述:电加热器(电热贴)}
		
		\begin{block}{电加热器(电热贴)的工作原理}
			电加热器(电热贴)通过电能转化为热能来实现加热,其工作原理基于焦耳定律$Q = I^2 R t$。
		\end{block}
		
		\begin{block}{加热功率的测量}
			使用电压表并联连接在电加热器两端,读取电压值;使用电流表串联连接在电路中,读取电流值;根据测得的电压和电流,使用公式$P = VI$计算加热功率。
			
			或者采用使用欧姆表测量电加热器的电阻,使用测得的电阻和电流计算$P = I^2 R$。
			
			对于电流电压随时间变化的情况,计算一段时间的总热功可以利用\textcolor{c4}{积分}来实现。
		\end{block}
		
	\end{frame}
	
	\begin{frame}
		\frametitle{实验原理概述:PID控温I}
		\begin{block}{控温算法}
			
			\begin{table}[h]
				\centering
				%\label{tab:tab1}
				%\caption{高级控温算法的比较}
				\resizebox{\textwidth}{!}{%
				\begin{tabular}{|c|c|c|}
					\hline
					控温算法 & 优点 & 缺点 \\
					\hline
					\textcolor{c4}{PID 控制} & 简单、易实现 & 性能有限,难以处理复杂系统 \\
					\hline
					模糊控制 & 不需要精确模型,适应性强 & 规则设计复杂,性能依赖于规则质量 \\
					\hline
					模型预测控制 & 最优性能,多变量处理 & 计算量大,依赖系统模型 \\
					\hline
					自适应控制 & 参数自调整,适应性强 & 实现复杂,收敛性问题 \\
					\hline
					神经网络控制 & 非线性处理能力强 & 训练复杂,数据需求大 \\
					\hline
					最优控制 & 理论性能优 & 实现复杂,计算量大 \\
					\hline
				\end{tabular}
			}				
			\end{table}
			
		\end{block}
	\end{frame}
	
	\begin{frame}
		\frametitle{实验原理概述:PID控温II}
		
		\begin{block}{先考虑采用最经典的PID控温}
			\begin{itemize}
				\small\item PID(Proportional-Integral-Derivative)控制是一种用于温度控制的经典算法,通过对误差的比例、积分和微分进行计算和调整,精确控制加热器的输出,从而实现温度的稳定控制。PID 控制器通常由三个部分组成:比例(P),积分(I),和微分(D)。
				
				\small\item PID 控温的控制量 $u(t)$ 可以表示为:$$u(t) = K_p e(t) + K_i \int_0^t e(\tau) d\tau + K_d \frac{d e(t)}{dt}$$
				
				\item \small 比例控制$P_{\text{out}} = K_p e(t)$直接与当前误差成比例,调整系统响应速度;积分控制$I_{\text{out}} = K_i \int_0^t e(\tau) d\tau$对误差进行积分,消除稳态误差,增强系统的长期精度;微分控制$D_{\text{out}} = K_d \frac{d e(t)}{dt}$对误差进行微分,预测误差变化趋势,减小超调和振荡。
			\end{itemize}			
		\end{block}
		
	\end{frame}
	
	\begin{frame}
		\frametitle{实验原理概述:PID控温III}
		
		\begin{block}{离散形式的 PID 控制算法}
			在实际应用中,PID 控制通常以\textcolor{c4}{离散形式}实现。离散 PID 控制算法如下:			
			\begin{myhighlight}
				\small\[
				u[k] = u[k-1] + K_p (e[k] - e[k-1]) + K_i e[k] + K_d \left( \frac{e[k] - e[k-1]}{T} \right)
				\]
			\end{myhighlight}
			其中:
			\begin{itemize}
				\item $u[k]$ 是第 $k$ 次采样时的控制输出;
				\item $e[k]$ 是第 $k$ 次采样时的误差;
				\item $T$ 是采样周期。
			\end{itemize}
		\end{block}
		
	\end{frame}
	
	% ---
	
	
	% 第二章
	\section{实验方案}
	
	% 章目录
	\frame{\frametitle{Outline}\tableofcontents[currentsection]}
	
	% 内容
	\begin{frame}
		\frametitle{实验方案:总体规划}
	\end{frame}
	
	\begin{frame}
		\frametitle{实验方案:搭建热机}
	\end{frame}
	
	\begin{frame}
		\frametitle{实验方案:温度控制}
	\end{frame}
	
	\begin{frame}
		\frametitle{实验方案:测量热机效率}
	\end{frame}
	
	\begin{frame}
		\frametitle{实验方案:优化}
	\end{frame}
	
	% ---
	
	
	% 致谢
	\begin{frame}
		
		\begin{center}
			{\Huge\calligra Thanks!}
		\end{center}
		
		\begin{figure}[htbp]
			\centering
			\includegraphics[width=0.7\textwidth]{name.png}
			%\caption{个人签名}
			%\label{fig:name}
		\end{figure}
		
	\end{frame}
	
	
	
	
	
	
	
%---------------------------------------------------------------------	
%---------------------------------------------------------------------
%---------------------------------------------------------------------
\end{document}